\Introduction

В настоящий момент потребность в надежной передаче данных между устройствами остается актуальной. Это 
связано с тем, что в любом реальном физическом канале связи присутствуют посторонние шумы, искажающие 
поступающие данные. Для устранения искажений и получения оригинальной информации в настоящее время 
широко используется помехоустойчивое кодирование информации. Исходный поток данных разбивается на 
порции данных, которые кодируются помехоустойчивым кодом и передаются в закодированном виде в канал 
связи. После принятия искаженных данных из канала выполняется процесс декодирования полученной 
информации. В случае обнаружения ошибок используются алгоритмы, позволяющие восстановить оригинальную 
информацию. При использовании кодирования к исходным данным добавляется избыточная информация, которая 
позволяет на этапе декодирования определить наличие ошибок и выполнить их устранение.

Пионером теории кодирования приято считать Клода Шеннона, опубликовавшего в 1948 году статью, в которой 
он определил понятие \textit{пропускной способности} канала. Было показано, что для каждого канала 
пропускная способность определяется числом $С$ и измеряется в битах в секунду. Затем было показано, что 
если скорость передачи данных $R$ (бит/сек) меньше, чем пропускная способность канала, то с 
использованием помехоустойчивых кодов можно добиться сколь угодно малой вероятности ошибки на выходе. В 
дальнейшем многие исследователи приложили свои усилия для отыскания классов кодов, которые позволяют 
получить малую вероятность ошибки. В частности, одно из направлений было посвящено изучению \textit{
блоковых кодов}. 

Первые блоковые коды были описаны Хеммингом в 1950 году. Он показал, что найденный им класс кодов 
позволяет строить коды, способные исправлять одиночные ошибки. Данное открытие стало прорывом в теории 
кодирования, и многие исследователи, воодушевленные работой Хемминга, продолжили поиски лучшего класса 
кодов. Однако, на протяжении десяти лет такой класс кодов получить не удалось. Но в 1960 году Боуз, 
Чоудхори и Хоквингем нашли класс кодов, позволяющих исправлять кратные ошибки. Полученные ими коды 
подучили название БЧХ-коды.

Другой класс кодов, позволяющий строить коды и кодировать ими информацию, получил название \textit{
сверточных кодов}. Развитие сверточных кодов заметно отличалось от развития теории блоковых кодов. 
Разницу можно почуствовать даже если сравнить подходы для нахождения хороших классов кодов. Так, при 
построении блоковых кодов и создании эффективных алгоритмов декодирования широко используются \textit{
алгебраические методы}. Со сверточными кодами ситуация иная. Например, хорошие сверточные коды были 
найдены с использованием вычислительной техники. Для этого выполнялся анализ большого числа кодов, а 
затем выбирались коды с лучшими характеристиками.

Другим отличительным свойством сверточных кодов является отсутствие методов декодирования, аналогичных 
алгебраическим методам исправления кратных ошибок. Наиболее используемым методом декодирования 
сверточных кодов является \textit{метод максимального правдоподобия}, на основе которого функционирует 
\textit{алгоритм Витерби}. Сверточное кодирование и декодирование с использованием алгоритма Витерби 
стало широко применяться на практике. Например, указанное кодирование применяется при передаче денных в 
космическом пространстве и при передаче денных из космического пространства на Землю и обратно. 
Причина, по которой данное кодирование столь распространено, состоит в относительной простоте 
реализации и в достижении выигрыша при декодировании.

Для улучшения характеристик системы передачи информации было предложено использовать \textit{каскадные 
коды}. В этом случае исходных поток данных после его разбиения на порции данных кодируется одним их 
блоковых кодов. Широко используемым кодом для данной цели является БЧХ-код. Далее полученное кодовое 
слово БЧХ-кода кодируется сверточным кодом. Полученное слово сверточного кода отправляется в канал. 
Таким образом два кодера работают в каскаде, откуда и произошло характерное название данного класса 
кодов. При декодировании каскадного кода выполняется ряд обратных кодированию процедур. Так, на первом 
этапе с использованием алгоритма Витерби происходит декодирование данных, поступивших из канала. В 
результате декодирования имеем кодовое слово БЧХ-кода, которое дополнительно декодируется БЧХ-
декодером. Особая эффективность такого способа кодирования состоит в том, что характерные свойства как 
БЧХ-кодов, так и сверточных кодов (и алгоритма Витерби) используются совместно, что позволяет достигать 
более высоких показателей, таких как вероятность ошибки на бит. Недостатком такого подхода является, 
пожалуй, относительная низкая скорость кода.