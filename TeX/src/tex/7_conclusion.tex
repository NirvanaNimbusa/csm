\Conclusion % заключение к отчёту
Представленная работа посвящена реализации и исследованию модели системы передачи информации при использовании каскадных кодов (коды 
БЧХ и сверточные коды). Указанная модель системы была реализована в виде набора независимых компонентов, каждый из которых был 
тщательно отлажен. По завершению этапа построения и реализации модели она была использована для имитационного моделирования процесса 
передачи данных изображения по системе для наглядной демонстрации результатов передачи данных и эффективности каскадного кодирования и 
для оценки зависимости $BER$ на приемнике от $BER$ в канале. Выполнялось имитационное моделирование при отключенном механизме БЧХ-
кодирования/декодирования для получения теоретических результатов и сравнения с ними результатов, полученных на практике в предыдущем 
случае.

Были получены результаты, которые показали, что сверточное кодирование и использование алгоритма Витерби при декодировании позволяют 
получать данные (изображения) приемлемого качества даже при значении $BER$ в канале, равном $BER=0.017011$. По сравнению с аналогичным 
значением в реальных физических каналах (не более $10^{-5}$) данный результат говорит о эффективности применения сверточного 
кодирования.

Использование БЧХ-кодирования позволяет восстанавливать информацию, которую не смог восстановить декодер Витерби. В реальных же 
системах, внешний декодер исправляет пакеты ошибок, которые генерирует декодер Витерби, в случае, если он не в состоянии исправить их. 
Таким образом совместное использование указанных способов кодирования позволяет повысить надежность передачи данных.

Другой результат связан с определением оптимального размера буфера накопленной последовательности узла решетки, которая применяется в 
декодере, работающем по алгоритму Витерби. Экспериментальные исследования показали, что действительно, увеличение размера указанного 
буфера позволяет получить выигрыш. Однако, чем больше буфер, тем меньше заметен выигрыш и при разработке реального устройства 
необходимо учитывать экономическую составляющую. Например, стоимость памяти, необходимой для реализации декодера Витерби при 
построении реальной системы передачи информации.

В целом, полученные результаты полностью соответствуют теории, что говорит о корректности реализации исследованных способов кодирования.

