\chapter{Описание программной модели}
Программная модель представляет собой консольное приложение, предназначенное для моделирования и анализа 
системы передачи данных при использовании каскадных кодов. Приложение имеет интерфейс командной строки и 
позволяет выполнять конфигурацию модели перед ее запуском путем указания параметров используемых кодов. В 
случае, если данные параметры не указываются, модель конфигурируется параметрами по умолчанию. Полный набор 
опций разработанной модели представлен в Листинге~\ref{lst_01}.

\begin{lstlisting}[caption={Вывод информации о доступных опциях}, label={lst_01}]
Administrator@DNAPC /cygdrive/d/SUAI/8sem/Alexeev/msvs10/scm_new/csm/build/bin/Release/app
$ ./app --h
Usage: csm [-uv] [-g <int>] [-l <int>] [-e <int>] [-d <int>] [-b <double>] [-p <string>] <file> [<file>]... [--help] [--version]
Program performs simulation of data transfer in communication system. The system is structured in the following manner:
[Transmitter]->[BCH encoder]->[Convolutional Encoder]->[Binary Symmetric Channel]->[Viterbi Decoder]->[BCH decoder]->[Receiver]
The following options are available:
  -g, --galois=<int>        define Evariste Galois field degree (between 2..20, default is 4)
  -l, --length=<int>        define BCH code length (default is 15)
  -e, --errors=<int>        define BCH code error correction property (default is 3)
  -d, --dbuffer=<int>       define internal buffer size of trellis node in Viterbi Decoder (default is 2 frames)
  -b, --ber=<double>        define channel Bit Error Rate (default is 1e-12)
  -p, --postfix=<string>    define out file postfix (default is 'transferred')
  <file>                    define input file(s)
  -u, --gui                 run under GUI
  -v, --verbose, --debug    verbose messages
  --help                    print this help and exit
  --version                 print version information and exit
\end{lstlisting}

В качестве исходного файла, предназначенного для его использования при моделировании процесса передачи 
данных, указывается потенциально любой файл. Следует учесть, что в случае, если исходный файл имеет размер, 
больший чем 1МБ, то процесс моделирования затянется на достаточное время. Это связано с тем, что при 
реализации данной модели акцент на оптимизацию по времени выполнения прочеса симуляции, размеру потребляемой 
памяти приложением, оценке производительности приложения в целом, не делался. Главной задачей было убедиться 
в работоспособности модели для получения с нее результатов моделирования.

\chapter{Запуск программной модели}
Генерируемая инормация после после окончания моделирования представлена в Листинге~\ref{lst_02}.
Характерной особенностью является автоматическое тестирвоание на предмет корректности принятых данных
на удаленной стороне.

\begin{lstlisting}[caption={Вывод информации после окончания моделирования}, label={lst_02}]
Administrator@DNAPC /cygdrive/d/SUAI/8sem/Alexeev/msvs10/scm_new/TeX/inc/img/chapter_4
$ ./app -b 0.0001 img_test_01.gif
Performing 'img_test_01.gif'
Progress: 100%

... transmitter stopped
... transmitter settings ...
data size: 156456
frame size: 5
... transmitter statistics ...
transmitted frames: 31292
transmitted bits: 156460

... receiver stopped
... receiver settings ...
data size: 156456
frame size: 5
... receiver statistics ...
received frames: 31292
valid frames: 31292 (100.00%)
invalid frames: 0 (0.00%)
received bits: 156460
valid bits: 156460 (100.00%)
invalid bits: 0 (0.00%)
BER: 0.000000
FER: 0.000000

... bch encoder stopped
... bch encoder settings ...
galois field degree: 4
code length: 15
information symbols quantity: 5
error correction property: 3
... bch encoder statistics ...
encoded frames: 31292
generated codewords: 31292
transmitted bits: 469380

... bch decoder stopped
... bch decoder settings ...
galois field degree: 4
code length: 15
information symbols quantity: 5
error correction property: 3
... bch decoder statistics ...
received bits: 469380
received bits (valid): 469380
received bits (corrupted): 0
received codewords: 31292
received codewords (valid): 31292
received codewords (corrupted): 0
generated frames: 31292

... cnv encoder stopped
... cnv encoder settings ...
quantity of registers: 2
codeword length: 30
... cnv encoder statistics ...
encoded frames: 31292
generated codewords: 31292
transmitted bits: 938760

... cnv decoder stopped
... cnv decoder settings ...
decoded sequence buffer size: 31
... cnv decoder statistics ...
delayed frames: 2
received bits: 938820
received bits (valid): 935880
received bits (corrupted): 98
received codewords: 31294
received codewords (valid): 31196
received codewords (corrupted): 98
generated frames: 31294

... channel bs stopped
... channel bs settings ...
BER: 0.000100
... channel bs statistics ...
bits transferred: 938820
bits corrupted: 98 (0.01%)
test: OK
\end{lstlisting}

\section{Используемый инcтрументарий и техника разработки}
При построении модели системы передачи информации использовалась интегрированная среда разработки Microsoft 
Visual Studio 2010. Разработка велась преимущественно на языке высокого уровня С с добавлением конструкций, 
характерных для языка С++. При построении модели применялось компонентно-ориентированное программирование --- 
парадигма программирования, ключевой фигурой которой является компонент. В качестве такого компонента 
выступает библиотека, реализующая необходимый функционал, и обладающая своим интерфейсом. Несмотря на широкое 
использование данного подхода при построении систем на С++, указанный подход являлся основой для построения 
системы в целом. В результате изменения в существующую систему вносятся путём создания новых компонентов в 
дополнение или в качестве замены ранее существующих. Особое значение было уделено интерфейсам компонентов и 
технике, позволяющей строить гибкие системы и использовать большинство широко распространенных паттернов 
проектирования в случае использования языка С.

При реализации модели автор старался применять разработку через тестирование (test-driven development, TDD) и 
модульное тестирование --- техники разработки программного обеспечения, которые основываются на повторении 
очень коротких циклов разработки. Ключевыми моментами являются частые релизы, рефакторинг, контроль 
архитектуры.

\section{Архитектура модели}
Модель реализуется в виде набора компонентов, представленных в виде статических библиотек, каждая из которых 
решает свою, четко обозначенную задачу. Так, в проекте используются библиотеки, предназначенные для 
кодирования/декодирования БЧХ-кодов, кодирования/декодирования сверточных кодов. Характерной особенностью 
является разнесение ответственности компонентов и разбиение кодеров и декодеров на кодеры переднего плана и 
кодеры заднего плана. Такая практика позволяет контролировать архитектуру отдельных компонентов библиотеки и 
гибко управлять интерфейсом компонентов. Более того, практика дает возможность гибкого модифицирования или же 
замены алгоритма работы компонента заднего плана. Помимо этого, имеются библиотеки, служащие для 
моделирования канала передачи данных, тестового окружения.
