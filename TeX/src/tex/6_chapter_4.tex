%%%%%%%%%%%%%%%%%%%%%%%%%%%%%%%%%%%%%%%%%%%%%%%%%%%%%%%%%%%%%%%%%%%%%%%%%%%%%%%%%%%%%%%%%%%
%%%%%%%%%%%%%%%%%%%%%%%%%%%%%%%%%%%%%%%%%%%%%%%%%%%%%%%%%%%%%%%%%%%%%%%%%%%%%%%%%%%%%%%%%%%
%%%%%%%%%%%%%%%%%%%%%%%%%%%%%%%%%%%%%%%%%%%%%%%%%%%%%%%%%%%%%%%%%%%%%%%%%%%%%%%%%%%%%%%%%%%
\chapter{Результаты моделирования}
Данный раздел содержит результаты моделирвоания передачи данных при использовании каскадного
кодирования. Моделирование выполнялось с использованием разработанного и протестированного
приложения. Основная цель моделирования --- получение наглядных результатов, демонстрирующих
эффективность каскадного кодирования и корреллирующих с теоретическими ожиданиями, а также
оценка частоты появления ошибок на удаленной стороне.

%%%%%%%%%%%%%%%%%%%%%%%%%%%%%%%%%%%%%%%%%%%%%%%%%%%%%%%%%%%%%%%%%%%%%%%%%%%%%%%%%%%%%%%%%%%
%%%%%%%%%%%%%%%%%%%%%%%%%%%%%%%%%%%%%%%%%%%%%%%%%%%%%%%%%%%%%%%%%%%%%%%%%%%%%%%%%%%%%%%%%%%
%%%%%%%%%%%%%%%%%%%%%%%%%%%%%%%%%%%%%%%%%%%%%%%%%%%%%%%%%%%%%%%%%%%%%%%%%%%%%%%%%%%%%%%%%%%
\section{Передача изображения по системе моделирования}
На Рис.~\ref{img:experimoriginal} -- \ref{img:experimcoded12} представлены исходное изображение, моделирование передачи которого
производилось на программной модели. На вход поступают данные изображения, затем они искажаются в канале
в зависимости от текущих параметров канала. Далее искаженные данные принимаются на удаленной стороне и
декодируются. Под каждым изображением указаны параметры модели. Нотация параметров коррелирует с обозначением
ключей приложения, реализующего программную модель.

\begin{figure}[h]
\begin{center}
\begin{minipage}[h]{0.4\linewidth}
\includegraphics[width=1\linewidth]{chapter_4/img_test_06.png}
\caption{Исходное изображение} %% подпись к рисунку
\label{img:experimoriginal} %% метка рисунка для ссылки на него
\end{minipage}
\hfill 
\begin{minipage}[h]{0.4\linewidth}
\includegraphics[width=1\linewidth]{chapter_4/img_test_06_transferred_g_4_l_15_e_3_d_31_b_0_000024.png}
\caption{g=4, l=15, e=3, d=31, b=0.000024}
\label{img:experimcoded2}
\end{minipage}
\end{center}

\begin{center}
\begin{minipage}[h]{0.4\linewidth}
\includegraphics[width=1\linewidth]{chapter_4/img_test_06_transferred_g_4_l_15_e_3_d_31_b_0_009844.png}
\caption{g=4, l=15, e=3, d=31, b=0.009844}
\label{img:experimcoded3}
\end{minipage}
\hfill 
\begin{minipage}[h]{0.4\linewidth}
\includegraphics[width=1\linewidth]{chapter_4/img_test_06_transferred_g_4_l_15_e_3_d_31_b_0_011813.png}
\caption{g=4, l=15, e=3, d=31, b=0.011813}
\label{img:experimcoded4}
\end{minipage}
\end{center}
\end{figure}

\begin{figure}[h]
\begin{center}
\begin{minipage}[h]{0.4\linewidth}
\includegraphics[width=1\linewidth]{chapter_4/img_test_06_transferred_g_4_l_15_e_3_d_31_b_0_029395.png}
\caption{g=4, l=15, e=3, d=31, b=0.029395}
\label{img:experimcoded9}
\end{minipage}
\hfill 
\begin{minipage}[h]{0.4\linewidth}
\includegraphics[width=1\linewidth]{chapter_4/img_test_06_transferred_g_4_l_15_e_3_d_31_b_0_035275.png}
\caption{g=4, l=15, e=3, d=31, b=0.035275}
\label{img:experimcoded10}
\end{minipage}
\end{center}

\begin{center}
\begin{minipage}[h]{0.4\linewidth}
\includegraphics[width=1\linewidth]{chapter_4/img_test_06_transferred_g_4_l_15_e_3_d_31_b_0_042329.png}
\caption{g=4, l=15, e=3, d=31, b=0.042329}
\label{img:experimcoded11}
\end{minipage}
\hfill 
\begin{minipage}[h]{0.4\linewidth}
\includegraphics[width=1\linewidth]{chapter_4/img_test_06_transferred_g_4_l_15_e_3_d_31_b_0_060954.png}
\caption{g=4, l=15, e=3, d=31, b=0.060954}
\label{img:experimcoded12}
\end{minipage}
\end{center}
\end{figure}

Из данных рисунков видно, что приемлeмое качество изображения сохраняется даже при $BER=0.017011$, что говорит
о эффективности методов каскадного кодирования. При использовании $BER=0.020413$ изображение начинает содержать
искажения, которые увеличиваются при увеличении $BER$ соответственно. В реальной жизни, как правило, уровень ошибок в канале связи
гораздо меньше указанных значений (обычно он меньше $10^{-5}$). Это означает, что принятое изображение будет в таком случае примлемого
качества и пользователь на удаленной стороне практически не заметит разницу в случае, если даже при декодировании не удалось
восстановить исходную информацию.

%%%%%%%%%%%%%%%%%%%%%%%%%%%%%%%%%%%%%%%%%%%%%%%%%%%%%%%%%%%%%%%%%%%%%%%%%%%%%%%%%%%%%%%%%%%
%%%%%%%%%%%%%%%%%%%%%%%%%%%%%%%%%%%%%%%%%%%%%%%%%%%%%%%%%%%%%%%%%%%%%%%%%%%%%%%%%%%%%%%%%%%
%%%%%%%%%%%%%%%%%%%%%%%%%%%%%%%%%%%%%%%%%%%%%%%%%%%%%%%%%%%%%%%%%%%%%%%%%%%%%%%%%%%%%%%%%%%
\section{Оценка частоты появления ошибок на удаленной стороне}
На Рис.~\ref{dependencies_all} показаны графики зависимостей логарифмическом масштабе $BER$ на приемнике от $BER$ в канале при условии,
что размер буфера накопленной последовательности в узле решетки декодера Витерби составляет 1-10 фреймов (1 фрейм состоит из 15 символов).

Для получения теоретических результатов и сравнения с ними результатов, полученных на практике был выполнен следующий
эксперимент. Выполнялось моедлирование передачи данных при отключенном БЧХ-кодировании/декодировании, то есть использовалось
только сверточное кодирование и декодирование по алгоритму Витерби. Данная конфигурация обусловлена тем, что достаточно
просто (для использовавшегося сверточного кодера) построить верхнюю оценку зависимости $BER$ на приемнике от $BER$ в канале.

График черного цвета показывает кривую,
получающуюся в результате теоретических вычислений. Из данного рисунка можно наблюдать, что теоретический результат
(верхняя граница) близок к результатам, полу-ченным благодаря моделированию.
\begin{figure}[h]
\begin{center}
\includegraphics[width=0.9\linewidth]{chapter_4/matlab_img/dependencies_all.png}
\caption{Зависимости BER на приемнике от BER в канале}
\label{dependencies_all}
\end{center}
\end{figure}

Данный результат полностью согласуется с теорией. Более того, по предыдущим иллюстрациям можно наблюдать, что
зависимость уменьшения $BER$ на приемнике от применения буфера накопленной последовательности в узле решетки
декодера Витерби большего размера сохраняется.